\documentclass{article}
\usepackage{graphicx}

\title{Learnings}


\begin{document}



\section{SMTP protocol}

The Simple Mail Transfer Protocol (SMTP) is a technical standard for transmitting electronic mail over a network. The main function of SMTP is to transfer data from the sender to the recipient. When a sender sends an email, SMTP opens a TCP (Transmission Control Protocol) connection, as it uses TCP as its transport protocol. The email client begins the email sending process with a specialized ``Hello'' command. The client then sends a series of commands containing the actual content of the email, such as the email header and email body. The server runs a program called MTA (Mail Transfer Agent), which checks the domain of the recipient's email address. After transferring the data, the connection is closed.

\section{SMTP Server}

An SMTP server is a mail server that can send and receive emails using the SMTP protocol. Various software programs run on an SMTP server. The Mail Submission Agent (MSA) receives emails from the email client. The Mail Transfer Agent (MTA) transfers email to the next server in the delivery chain. The Mail Delivery Agent (MDA) receives emails from MTAs and stores them in the recipient's mailbox.




\section {DTN}

Delay-Tolerant Networks (DTNs) are overlay networks constructed atop regional networks—networks characterized by homogeneous communication properties—and are designed to provide interoperability between these networks in environments where traditional network architectures and communication protocols perform poorly.  1 DTNs are distinguished by intermittent connectivity, long or variable delays, asymmetric data rates, and high error rates.  2 The concept originated in interplanetary networking, where deep space communications experience delays ranging from minutes to days, which cannot be managed by existing terrestrial networking technologies.  3 Since then, DTNs have been generalized to address communication challenges in various terrestrial and harsh environments, including wireless sensor networks and mobile ad hoc networks.
Traditional Internet protocols, such as Transmission Control Protocol/Internet Protocol (TCP/IP), assume continuous, bidirectional end-to-end paths between source and destination, short round-trip times, symmetric data rates, and low error rates.  2 These assumptions are often violated in DTN scenarios, where connectivity is intermittent and delays can be significant. TCP/IP fails to deliver messages to temporarily unavailable nodes and reports connection errors, whereas DTNs enable communication by employing a store-and-forward message switching approach that overcomes network interruptions and partitions.  1 The DTN architecture, as defined by the Delay-Tolerant Networking Research Group (DTNRG), introduces the bundle layer as an overlay that provides transparent communication among different regional networks and hides disconnection and delay from the application layer.

\section {Principles and Architecture of Delay-Tolerant Networks}
DTNs operate based on the store-carry-and-forward paradigm, where nodes store data bundles persistently, carry them as they move, and forward them opportunistically when a suitable relay is encountered, rather than relying on continuous end-to-end connectivity.  4 1 The Bundle Protocol, standardized in RFC 5050, defines the basic data unit transmitted across DTN nodes as bundles—variable-length protocol data units containing all necessary information for transaction completion, including protocol and authentication data.  5 Bundles consist of a sequence of blocks, including a primary block with processing information such as source and destination nodes, creation and expiration times, and size, as well as a payload block for the actual transmitted data; additional extension blocks may also be present.
The DTN architecture is implemented as an overlay network, with the bundle layer positioned between the application and transport layers. This positioning enables transparent communication among heterogeneous regional networks and hides disconnection and delay from the application layer.  1 Convergence Layer Adapters (CLAs) interface the bundle layer with various underlying transport protocols, such as TCP, User Datagram Protocol (UDP), and Licklider Transmission Protocol (LTP), allowing each DTN node to deploy different CLAs for reliable bundle transfer across heterogeneous network segments.  5 DTNs support interoperability by encapsulating application data into bundles and using Endpoint Identifiers (EIDs) for naming, which differ from traditional IP addressing.
Custody transfer is a mechanism in the bundle layer that provides end-to-end reliability by employing node-to-node retransmission; a node stores a bundle until another node accepts custody or the bundle’s time-to-live expires. Acknowledgments and retransmissions ensure data integrity.  1 Unlike traditional Internet protocols, which assume continuous, bidirectional end-to-end paths and rely on packet switching, DTNs are designed to operate under intermittent connectivity, long or variable delays, asymmetric data rates, and high error rates, fundamentally impacting network design and requiring persistent storage at nodes for reliable communication.  4 5 Security and reliability in DTNs depend on upper-layer mechanisms, as the Bundle Protocol itself does not provide error detection or correction capabilities.


\end{document}
